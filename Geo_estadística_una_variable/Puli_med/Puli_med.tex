\PassOptionsToPackage{unicode=true}{hyperref} % options for packages loaded elsewhere
\PassOptionsToPackage{hyphens}{url}
%
\documentclass[ignorenonframetext,]{beamer}
\usepackage{pgfpages}
\setbeamertemplate{caption}[numbered]
\setbeamertemplate{caption label separator}{: }
\setbeamercolor{caption name}{fg=normal text.fg}
\beamertemplatenavigationsymbolsempty
% Prevent slide breaks in the middle of a paragraph:
\widowpenalties 1 10000
\raggedbottom
\setbeamertemplate{part page}{
\centering
\begin{beamercolorbox}[sep=16pt,center]{part title}
  \usebeamerfont{part title}\insertpart\par
\end{beamercolorbox}
}
\setbeamertemplate{section page}{
\centering
\begin{beamercolorbox}[sep=12pt,center]{part title}
  \usebeamerfont{section title}\insertsection\par
\end{beamercolorbox}
}
\setbeamertemplate{subsection page}{
\centering
\begin{beamercolorbox}[sep=8pt,center]{part title}
  \usebeamerfont{subsection title}\insertsubsection\par
\end{beamercolorbox}
}
\AtBeginPart{
  \frame{\partpage}
}
\AtBeginSection{
  \ifbibliography
  \else
    \frame{\sectionpage}
  \fi
}
\AtBeginSubsection{
  \frame{\subsectionpage}
}
\usepackage{lmodern}
\usepackage{amssymb,amsmath}
\usepackage{ifxetex,ifluatex}
\usepackage{fixltx2e} % provides \textsubscript
\ifnum 0\ifxetex 1\fi\ifluatex 1\fi=0 % if pdftex
  \usepackage[T1]{fontenc}
  \usepackage[utf8]{inputenc}
  \usepackage{textcomp} % provides euro and other symbols
\else % if luatex or xelatex
  \usepackage{unicode-math}
  \defaultfontfeatures{Ligatures=TeX,Scale=MatchLowercase}
\fi
% use upquote if available, for straight quotes in verbatim environments
\IfFileExists{upquote.sty}{\usepackage{upquote}}{}
% use microtype if available
\IfFileExists{microtype.sty}{%
\usepackage[]{microtype}
\UseMicrotypeSet[protrusion]{basicmath} % disable protrusion for tt fonts
}{}
\IfFileExists{parskip.sty}{%
\usepackage{parskip}
}{% else
\setlength{\parindent}{0pt}
\setlength{\parskip}{6pt plus 2pt minus 1pt}
}
\usepackage{hyperref}
\hypersetup{
            pdftitle={Pulimiento de medianas},
            pdfborder={0 0 0},
            breaklinks=true}
\urlstyle{same}  % don't use monospace font for urls
\newif\ifbibliography
\usepackage{color}
\usepackage{fancyvrb}
\newcommand{\VerbBar}{|}
\newcommand{\VERB}{\Verb[commandchars=\\\{\}]}
\DefineVerbatimEnvironment{Highlighting}{Verbatim}{commandchars=\\\{\}}
% Add ',fontsize=\small' for more characters per line
\usepackage{framed}
\definecolor{shadecolor}{RGB}{248,248,248}
\newenvironment{Shaded}{\begin{snugshade}}{\end{snugshade}}
\newcommand{\AlertTok}[1]{\textcolor[rgb]{0.94,0.16,0.16}{#1}}
\newcommand{\AnnotationTok}[1]{\textcolor[rgb]{0.56,0.35,0.01}{\textbf{\textit{#1}}}}
\newcommand{\AttributeTok}[1]{\textcolor[rgb]{0.77,0.63,0.00}{#1}}
\newcommand{\BaseNTok}[1]{\textcolor[rgb]{0.00,0.00,0.81}{#1}}
\newcommand{\BuiltInTok}[1]{#1}
\newcommand{\CharTok}[1]{\textcolor[rgb]{0.31,0.60,0.02}{#1}}
\newcommand{\CommentTok}[1]{\textcolor[rgb]{0.56,0.35,0.01}{\textit{#1}}}
\newcommand{\CommentVarTok}[1]{\textcolor[rgb]{0.56,0.35,0.01}{\textbf{\textit{#1}}}}
\newcommand{\ConstantTok}[1]{\textcolor[rgb]{0.00,0.00,0.00}{#1}}
\newcommand{\ControlFlowTok}[1]{\textcolor[rgb]{0.13,0.29,0.53}{\textbf{#1}}}
\newcommand{\DataTypeTok}[1]{\textcolor[rgb]{0.13,0.29,0.53}{#1}}
\newcommand{\DecValTok}[1]{\textcolor[rgb]{0.00,0.00,0.81}{#1}}
\newcommand{\DocumentationTok}[1]{\textcolor[rgb]{0.56,0.35,0.01}{\textbf{\textit{#1}}}}
\newcommand{\ErrorTok}[1]{\textcolor[rgb]{0.64,0.00,0.00}{\textbf{#1}}}
\newcommand{\ExtensionTok}[1]{#1}
\newcommand{\FloatTok}[1]{\textcolor[rgb]{0.00,0.00,0.81}{#1}}
\newcommand{\FunctionTok}[1]{\textcolor[rgb]{0.00,0.00,0.00}{#1}}
\newcommand{\ImportTok}[1]{#1}
\newcommand{\InformationTok}[1]{\textcolor[rgb]{0.56,0.35,0.01}{\textbf{\textit{#1}}}}
\newcommand{\KeywordTok}[1]{\textcolor[rgb]{0.13,0.29,0.53}{\textbf{#1}}}
\newcommand{\NormalTok}[1]{#1}
\newcommand{\OperatorTok}[1]{\textcolor[rgb]{0.81,0.36,0.00}{\textbf{#1}}}
\newcommand{\OtherTok}[1]{\textcolor[rgb]{0.56,0.35,0.01}{#1}}
\newcommand{\PreprocessorTok}[1]{\textcolor[rgb]{0.56,0.35,0.01}{\textit{#1}}}
\newcommand{\RegionMarkerTok}[1]{#1}
\newcommand{\SpecialCharTok}[1]{\textcolor[rgb]{0.00,0.00,0.00}{#1}}
\newcommand{\SpecialStringTok}[1]{\textcolor[rgb]{0.31,0.60,0.02}{#1}}
\newcommand{\StringTok}[1]{\textcolor[rgb]{0.31,0.60,0.02}{#1}}
\newcommand{\VariableTok}[1]{\textcolor[rgb]{0.00,0.00,0.00}{#1}}
\newcommand{\VerbatimStringTok}[1]{\textcolor[rgb]{0.31,0.60,0.02}{#1}}
\newcommand{\WarningTok}[1]{\textcolor[rgb]{0.56,0.35,0.01}{\textbf{\textit{#1}}}}
\usepackage{longtable,booktabs}
\usepackage{caption}
% These lines are needed to make table captions work with longtable:
\makeatletter
\def\fnum@table{\tablename~\thetable}
\makeatother
\setlength{\emergencystretch}{3em}  % prevent overfull lines
\providecommand{\tightlist}{%
  \setlength{\itemsep}{0pt}\setlength{\parskip}{0pt}}
\setcounter{secnumdepth}{0}

% set default figure placement to htbp
\makeatletter
\def\fps@figure{htbp}
\makeatother


\title{Pulimiento de medianas}
\author{}
\date{\vspace{-2.5em}}

\begin{document}
\frame{\titlepage}

\hypertarget{simulaciuxf3n-de-un-campo-aleatorio.}{%
\section{Simulación de un campo
aleatorio.}\label{simulaciuxf3n-de-un-campo-aleatorio.}}

\begin{frame}{Cargar librerias}
\protect\hypertarget{cargar-librerias}{}

Lista de librerías con link a la documentación.

\begin{itemize}
\tightlist
\item
  \href{https://cran.r-project.org/web/packages/gstat/gstat.pdf}{gstat}
\item
  \href{https://cran.r-project.org/web/packages/sp/sp.pdf}{sp}
\end{itemize}

Lista de librerías con link a la documentación.

\end{frame}

\begin{frame}[fragile]{Grilla de las ubicaciones espaciales.}
\protect\hypertarget{grilla-de-las-ubicaciones-espaciales.}{}

Encabezado coordenadas

\begin{Shaded}
\begin{Highlighting}[]
\NormalTok{knitr}\OperatorTok{::}\KeywordTok{kable}\NormalTok{(}\KeywordTok{head}\NormalTok{(coordenadas), }\DataTypeTok{label =} \StringTok{"Encabezado coordenadas"}\NormalTok{)}
\end{Highlighting}
\end{Shaded}

\begin{longtable}[]{@{}rr@{}}
\toprule
X & Y\tabularnewline
\midrule
\endhead
0.0000000 & 0.0\tabularnewline
0.3333333 & 0.0\tabularnewline
0.6666667 & 0.0\tabularnewline
1.0000000 & 0.0\tabularnewline
0.0000000 & 0.2\tabularnewline
0.3333333 & 0.2\tabularnewline
\bottomrule
\end{longtable}

\end{frame}

\begin{frame}[fragile]{Definición de objeto VGM}
\protect\hypertarget{definiciuxf3n-de-objeto-vgm}{}

\begin{itemize}
\tightlist
\item
  \href{https://cran.r-project.org/web/packages/gstat/gstat.pdf\#page=73}{vgm}
\end{itemize}

\begin{Shaded}
\begin{Highlighting}[]
\NormalTok{vario <-}\StringTok{ }\KeywordTok{vgm}\NormalTok{(}\DecValTok{10}\NormalTok{, }\CommentTok{# Punto de silla}
             \StringTok{"Exp"}\NormalTok{, }\CommentTok{# Modelo, ver documentación}
             \FloatTok{0.5}\NormalTok{)  }\CommentTok{# Rango}
\KeywordTok{print}\NormalTok{(vario)}
\end{Highlighting}
\end{Shaded}

\begin{verbatim}
##   model psill range
## 1   Exp    10   0.5
\end{verbatim}

\end{frame}

\begin{frame}[fragile]{Matriz de varianza dadas coordenadas.}
\protect\hypertarget{matriz-de-varianza-dadas-coordenadas.}{}

\begin{itemize}
\tightlist
\item
  \href{https://cran.r-project.org/web/packages/gstat/gstat.pdf\#page=78}{vgmArea}
\item
  \href{https://cran.r-project.org/web/packages/sp/sp.pdf\#page=16}{coordinates}
\end{itemize}

\begin{Shaded}
\begin{Highlighting}[]
\KeywordTok{coordinates}\NormalTok{(coordenadas) <-}\StringTok{ }\ErrorTok{~}\NormalTok{X }\OperatorTok{+}\StringTok{ }\NormalTok{Y}
\KeywordTok{class}\NormalTok{(coordenadas) }\CommentTok{# Cambio de objedto dataframe a sp}
\end{Highlighting}
\end{Shaded}

\begin{verbatim}
## [1] "SpatialPoints"
## attr(,"package")
## [1] "sp"
\end{verbatim}

\begin{Shaded}
\begin{Highlighting}[]
\NormalTok{cov_mat <-}\StringTok{ }\KeywordTok{vgmArea}\NormalTok{(coordenadas, }\CommentTok{# Matriz de ubiaciones SP}
        \DataTypeTok{vgm =}\NormalTok{ vario) }\CommentTok{# VGM object}

\KeywordTok{print}\NormalTok{(}\KeywordTok{dim}\NormalTok{(cov_mat))}
\end{Highlighting}
\end{Shaded}

\begin{verbatim}
## [1] 24 24
\end{verbatim}

\end{frame}

\begin{frame}[fragile]{Simulación.}
\protect\hypertarget{simulaciuxf3n.}{}

\begin{Shaded}
\begin{Highlighting}[]
\NormalTok{mu  <-}\StringTok{ }\KeywordTok{rep}\NormalTok{(}\DecValTok{0}\NormalTok{, n_x }\OperatorTok{*}\StringTok{ }\NormalTok{n_y) }\CommentTok{# Media del proceso}
\NormalTok{simu <-}\StringTok{ }\KeywordTok{rmvnorm}\NormalTok{(}\DecValTok{1}\NormalTok{,}
                \DataTypeTok{mean =}\NormalTok{ mu,}
                \DataTypeTok{sigma =}\NormalTok{ cov_mat)}
\KeywordTok{print}\NormalTok{(simu)}
\end{Highlighting}
\end{Shaded}

\begin{verbatim}
##           [,1]      [,2]      [,3]      [,4]      [,5]      [,6]      [,7]      [,8]      [,9]      [,10]     [,11]
## [1,] -4.282744 -2.500688 -2.473144 -4.069976 -3.339916 -1.253902 -2.333074 -4.304353 -3.287125 -0.8492653 0.4972164
##          [,12]     [,13]     [,14]    [,15]     [,16]     [,17]     [,18]    [,19]    [,20]     [,21]      [,22]
## [1,] -5.253496 -2.111975 -2.599939 2.898355 -0.666937 -2.606326 -0.270931 1.629065 3.832959 -2.470549 -0.6196649
##         [,23]  [,24]
## [1,] 3.923153 3.8529
\end{verbatim}

\begin{Shaded}
\begin{Highlighting}[]
\NormalTok{data <-}\StringTok{ }\KeywordTok{cbind}\NormalTok{(coordenadas}\OperatorTok{@}\NormalTok{coords, }\DataTypeTok{Simula =} \KeywordTok{t}\NormalTok{(simu))}
\KeywordTok{names}\NormalTok{(data) <-}\StringTok{ }\KeywordTok{c}\NormalTok{(}\StringTok{"X"}\NormalTok{, }\StringTok{"Y"}\NormalTok{, }\StringTok{"Var"}\NormalTok{)}
\KeywordTok{print}\NormalTok{(}\KeywordTok{head}\NormalTok{(data))}
\end{Highlighting}
\end{Shaded}

\begin{verbatim}
##           X   Y          
## 1 0.0000000 0.0 -4.282744
## 2 0.3333333 0.0 -2.500688
## 3 0.6666667 0.0 -2.473144
## 4 1.0000000 0.0 -4.069976
## 5 0.0000000 0.2 -3.339916
## 6 0.3333333 0.2 -1.253902
\end{verbatim}

\end{frame}

\end{document}
